\documentclass[10pt,a4paper]{article}

\usepackage[utf8]{inputenc}
\usepackage[english]{babel}
\usepackage[english]{isodate}
\usepackage[parfill]{parskip}
\usepackage{listings}
\usepackage{color}

\definecolor{gray}{rgb}{0.9,0.9,0.9}
\definecolor{white}{rgb}{1.0,1.0,1.0}
\lstdefinestyle{bash}{
backgroundcolor=\color{gray},
basicstyle=\ttfamily,
showstringspaces=false,
language=bash
}

\lstdefinestyle{file}{
backgroundcolor=\color{white},
basicstyle=\ttfamily,
showstringspaces=false,
frame=lines
}

\lstset{style=bash}

\begin{document}

\title{Electrostatic Tools Manual}
\author{Oleg I. Titov and Dmitry A. Shulga}

\maketitle
\newpage

\tableofcontents
\newpage

\section{About}
Electrostatic Tools is a program package that aims to assist in developing of
next-generation molecular electrostatic models with an enhanced molecular 
electrostatic potential (MEP) anisotropy. It provides the researcher with tools to
fit and for work with customizable atomic multipole moments (up to a quadrupole) and
extra-point charges.
The models obtained are intended to use for description of an electrostatic part of 
such highly anisotropic moieties as oxygens, nitrogens with lone pairs and heavy 
halogens (Cl, Br, I) for halogen bonding.
\newpage

\section{Installation}
The basic installation requires a functioning C++ compiler and development versions of all
prerequisites installed. Electrostatic Tools currently depends on:
\begin{itemize}
\item \textbf{CMake} is used to control the build and installation process
\item \textbf{OpenBabel} library is used to read and write molecules in common 
chemical formats and perform SMARTS pattern matching
\item \textbf{Eigen3} library provides linear algebra solvers
\item \textbf{boost} is also required (with \lstinline{program_options} binaries compiled)
\item \textit{optional} \textbf{SWIG} to build script bindings. Currently only 
\textbf{Perl} and \textbf{Python} are supported
\end{itemize}

Common procedure for all cmake-based packages can be applied:
\begin{lstlisting}[language=bash]
tar -xf electrostatic-tools-0.4.0.tar.bz2
cd electrostatic-tools-0.4.0
mkdir build
cd build
cmake ..
make
make install
\end{lstlisting}

One can be interested in the following options that can be passed to
cmake:
\begin{itemize}
\item
\lstinline{-DCMAKE_INSTALL_PREFIX=__prefix__}
This tells cmake to install the package in the \lstinline{__prefix__} directory. Most 
would like to use something like \lstinline{~/el_tools}. 
The default is \lstinline{/usr/local}.
\item
\lstinline{-DCMAKE_BUILD_TYPE=Release}
This will change build configuration to a \lstinline{Release} version meaning fast 
optimized executables. The other useful option is \lstinline{RelWithDebInfo} 
meaning the release-optimized executables with a debugging info.
\item
\lstinline{-DET_NO_mult_fitter=ON/OFF}, 
\lstinline{-DET_NO_ep_fitter=ON/OFF} and 
\lstinline{-DET_NO_mmol2mol=ON/OFF}
These switches turn off compilation and installation of the corresponding programs.
The defaults are \lstinline{OFF}
\item
\lstinline{-DET_PERL_BINDINGS=ON/OFF} and
\lstinline{-DET_PYTHON_BINDINGS=ON/OFF}
These switches turn off compilation and installation of the corresponding language bindings.
\item
\lstinline{-DET_PERL_LIBDIR} and
\lstinline{-DET_PYTHON_LIBDIR}
Manually specify the destination of language bindings.
\item
\lstinline{-DET_PERL_INCLUDE_PATH},
\lstinline{-DET_PERL_LIBRARY},
\lstinline{-DET_PYTHON_INCLUDE_PATH} and
\lstinline{-DET_PYTHON_LIBRARY}
can be used to override \lstinline{cmake}'s guesses.
\end{itemize}
For example if you want to install the release-optimized package in your home directory, 
you don't need Perl bindings and you want to install Python bindings in subfolder in your home 
directory, you should call:
\begin{lstlisting}
cmake -DCMAKE_BUILD_TYPE=Release                   \
      -DCMAKE_INSTALL_PREFIX=~/elec_tools          \
      -DET_PERL_BINDINGS=OFF                       \
      -DET_PYTHON_LIBDIR=~/elec_tools/lib          \
      ..
\end{lstlisting}

To temporarly add Electrostatic Tools to a \lstinline{$PATH} variable invoke:
\begin{lstlisting}[language=bash]
export PATH=$PATH:__prefix__/bin
\end{lstlisting}
for bash shell, where \lstinline{__prefix__} is your installation prefix, 
or for csh shell:
\begin{lstlisting}[language=csh]
setenv PATH $PATH:__prefix__/bin
\end{lstlisting}

To permanently add Electrostatic Tools to \lstinline{$PATH} invoke:
\begin{lstlisting}[language=bash]
echo "export PATH=$PATH:__prefix__/bin" >> ~/.bashrc
\end{lstlisting}
for bash shell, where \lstinline{__prefix__} is your installation prefix, 
or for csh shell:
\begin{lstlisting}[language=csh]
echo "setenv PATH $PATH:__prefix__/bin" >> ~/.cshrc
\end{lstlisting}

\newpage
\section{Tutorial}
In this section we will discuss the usage of our programs. Files used in this tutorial
are shipped with the source package in an \lstinline{examples} directory. We assume that 
the package is compiled, installed and added to \lstinline{$PATH}. 
Every program in Electrostatic Tools 
has a build-in help available with a \lstinline{--help} switch.

\subsection{Multipole Fit}
To perform a multipole fit you will need a molecule structure in any common file formats
supported by OpenBabel and a reference electrostatic potential in \lstinline{.esp} 
format (used for a RESP-charges fitting within AmberTools package). 
The sample files with a phenylbromide molecule (\lstinline{phbr.mol2}) and the reference 
RHF/6-31G* potential 
(\lstinline{phbr.esp}) are located in the \lstinline{examples} directory of the source tree. 
We'll make a temporary working directory and copy the files there assuming that the 
source code was
untarred in home directory.
\begin{lstlisting}[language=bash]
mkdir ~/tutorial
cd ~/tutorial
cp ~/electrostatic-tools-0.4.0/examples/phbr.mol2 .
cp ~/electrostatic-tools-0.4.0/examples/phbr.esp .
\end{lstlisting}
To perform a simple fit with default parameters invoke:
\begin{lstlisting}[language=bash]
mult_fitter phbr.mol2 phbr.esp
\end{lstlisting}
In case everything went well no console output appears. This command prefrormed
the fit of atomic charges and multipoles of the phenylbromide molecule to the reference MEP 
with
default settings. Two files should appear in the working directory: \lstinline{phbr.mmol} 
and \lstinline{phbr.log}.
The first one contains the results - a phenylbromide molecule with multipoles, the second
one is a log with an additional information describing the fit process. 
Your \lstinline{phbr.mmol} file should look like the following:
\begin{lstlisting}[style=file]
Molecule:
/* Atom 1 Br */
Atom: 35 ( 0.9991, 0.0803, 0.1295 )
Multipoles: ( 0.9991, 0.0803, 0.1295 )
  Monopole: -0.18347
  Quadrupole: ( -0.81548, 0, 0, 0, -0.81548, 0, 0, 0, 1.631 )

/* Atom 2 C */
Atom: 6 ( 2.882, 0.1014, -0.0402 )
Multipoles: ( 2.882, 0.1014, -0.0402 )
  Monopole: 0.27616

/* Atom 3 C */
Atom: 6 ( 3.5573, 1.3195, -0.1026 )
Multipoles: ( 3.5573, 1.3195, -0.1026 )
  Monopole: -0.26482

/* Atom 4 C */
Atom: 6 ( 4.9466, 1.3334, -0.2343 )
Multipoles: ( 4.9466, 1.3334, -0.2343 )
  Monopole: -0.11137

/* Atom 5 C */
Atom: 6 ( 5.6536, 0.132, -0.3014 )
Multipoles: ( 5.6536, 0.132, -0.3014 )
  Monopole: -0.15636

/* Atom 6 C */
Atom: 6 ( 4.9736, -1.0845, -0.235 )
Multipoles: ( 4.9736, -1.0845, -0.235 )
  Monopole: -0.11137

/* Atom 7 C */
Atom: 6 ( 3.5842, -1.1018, -0.1044 )
Multipoles: ( 3.5842, -1.1018, -0.1044 )
  Monopole: -0.26482

/* Atom 8 H */
Atom: 1 ( 3.0135, 2.2588, -0.0508 )
Multipoles: ( 3.0135, 2.2588, -0.0508 )
  Monopole: 0.18885

/* Atom 9 H */
Atom: 1 ( 5.478, 2.2803, -0.2855 )
Multipoles: ( 5.478, 2.2803, -0.2855 )
  Monopole: 0.14652

/* Atom 10 H */
Atom: 1 ( 6.7358, 0.1442, -0.4058 )
Multipoles: ( 6.7358, 0.1442, -0.4058 )
  Monopole: 0.14529

/* Atom 11 H */
Atom: 1 ( 5.5259, -2.0194, -0.2859 )
Multipoles: ( 5.5259, -2.0194, -0.2859 )
  Monopole: 0.14652

/* Atom 12 H */
Atom: 1 ( 3.0612, -2.0529, -0.0547 )
Multipoles: ( 3.0612, -2.0529, -0.0547 )
  Monopole: 0.18885

Bond: 1 - 2 : 1
Bond: 2 - 3 : a
Bond: 3 - 4 : a
Bond: 4 - 5 : a
Bond: 5 - 6 : a
Bond: 6 - 7 : a
Bond: 2 - 7 : a
Bond: 3 - 8 : 1
Bond: 4 - 9 : 1
Bond: 5 - 10 : 1
Bond: 6 - 11 : 1
Bond: 7 - 12 : 1

Orient-rules:
rule:   z       "*"
rule:   a       "[!#99]~[!#99]"
rule:   a       "[!#99]#[!#99]"
rule:   b       "[!#99]~[!#99]=,@[!#99]"
rule:   b       "[!#99]=[!#99]~[!#99]"
rule:   c       "[!#99](~[!#99])~[!#99]"
rule:   c       "[!#99](=,@[!#99])~[!#99]"
rule:   d       "[!#99]([!#99])([!#99])[!#99]"
rule:   b       "[!#99](=[!#99])([!#99])[!#99]"
rule:   z       "[!#99]([!#99])([!#99])([!#99])[!#99]"
rule:   a       "[!#99](=[!#99])=[!#99]"
rule:   a       "[!#99](=[!#99!#6])=[!#99]"
rule:   b       "[!#99](~[!#99])(=[!#99])=[!#99]"
rule:   a       "[#99]1[!#99]#[!#99]1"
rule:   e       "[#99]1[!#99A]([!#99])=,@[!#99A]1"
rule:   e       "[#99]1[!#99]-,=,@[!#99]1"

/*
Fitted 15 parameters with 7 constraints (4617 reference points)
RMSD: 0.6277 kcal/mol
*/
\end{lstlisting}
It consists of three sections: the molecule description, the rules for orientation of the
multipoles and a comment. 

We support C and C-style comments (\lstinline[language=C]{/* comment */} 
\lstinline[language=C]{// comment till the end of line}) in our custom 
file formats, so it
is possible to store any relevant information directly in the file with a molecule.
\lstinline{mult_fitter} saves a fitted MEP description error and a number of parameters 
with a number of applied constraints.

The molecule section consists of a description of atoms and bonds. The bond format is 
intuitive. The atoms are
stored as pairs of nucleus charges with coordinates in Angstroms. Every atom may have a
single multipole expansion associated with it. The multipoles values are printed in atomic
units. The quadrupole matrix is printed in a row-by-row manner on a single line. Note that
the multipoles are shown in principal axes, so they can be easily analysed.

The contents of the \lstinline{phbr.log} file should be as the following:
\begin{lstlisting}[style=file]
Multipole fitter v 0.4.0
MEP Fitter initialized.
Forcing tolopogical equivalency: 1
Tolopogical information will be recalculated: 0
Multipole refit requested: 0
Molecule with 12 atoms loaded.
19 atom and 0 group multipole positioning rules loaded.
4617 points of field loaded.
0 dummy atoms added.
Created distance matrix 4617x15
Using SVD fitter.
Created constraints matrix 7x15
Fitting
Fit completed.
Fitted 15 parameters with 7 constraints (4617 reference points)
RMSD: 0.6277 kcal/mol
----------------- Constraints -----------------
Parameters: Z1  Qxx1    Qyy1    Qzz1    Z2      Z3      Z4      Z5      Z6      Z7      Z8      Z9      Z10     Z11     Z12
Constraints matrix (B):
 0  1 -1  0  0  0  0  0  0  0  0  0  0  0  0
 0  0  0  0  0  0  0  0  0  0  1  0  0  0 -1
 0  0  0  0  0  0  0  0  0  0  0  1  0 -1  0
 0  0  0  0  0  1  0  0  0 -1  0  0  0  0  0
 0  0  0  0  0  0  1  0 -1  0  0  0  0  0  0
 0  1  1  1  0  0  0  0  0  0  0  0  0  0  0
 1  0  0  0  1  1  1  1  1  1  1  1  1  1  1
Constraints vecctor (C):
0
0
0
0
0
0
0
----------------- SVD analysis (after constraints applied) -----------------
Condition number cutoff: 1e+07
Condition number of the system: 110.373
Singular values:
    5.88118
    2.25861
    1.44738
    1.21205
   0.482187
   0.189836
  0.0702794
  0.0532848
1.29054e-15
...
Parameters: Z1  Qxx1    Qyy1    Qzz1    Z2      Z3      Z4      Z5      Z6      Z7      Z8      Z9      Z10     Z11     Z12
Right singular vectors (V**T):
   0.516804  0.00163018  0.00163018 -0.00326036    0.260875    0.140279   -0.144239   -0.294506   -0.144239    0.140279
0.254092   -0.240601   -0.502233   -0.240601    0.254092
   0.642964 -0.00118616 -0.00118616  0.00237231    0.058724   -0.143181    -0.10988    0.149204    -0.10988   -0.143181
-0.238105   -0.206713    0.544868   -0.206713   -0.238105
...
\end{lstlisting}
Along with the details of the fit (including the actual constraints matrix) 
this file contains a valuable information provided by a SVD
fitter: the condition number of a model matrix and a list of singular values with right
singular vectors. In this particular case the first shows that the charges and 
the quadrupole were well-defined. The
last two ones help analyse the system if any problem with the fit stability occurs. The
''\lstinline{Parameters:}'' header and the ''\lstinline{V**T}'' matrix are tab separated 
so they can
be easily pasted in any spreadsheet processor for further analysis. 

If you try to rerun the calculation (note a different way to pass command line arguments) 
it will terminate with the following error:
\begin{lstlisting}[language=bash]
$mult_fitter -I phbr.mol2 -G phbr.esp -O phbr.mmol \
             -L phbr.log
Error: output molecule file exists. Aborting.
\end{lstlisting}
No Electrostatic Tools program will overwrite existing files to preserve any previous 
results in case
of a typo. To ignore this and overwrite the existing files pass a \lstinline{-f} key.

\subsection{Multipole Placement}
In the previuos example we used the default multipole placement policy. This policy can be
overriden by specifying a multipole placement rulefile. Let's assume we want to place a
symmetric quadrupole on the bromine and a dipole on the \textit{p}-hydrogen atoms. 
Create a custom file with the rules and perform the fit with these rules by invoking (note 
another way of passing the filenames):
\begin{lstlisting}[language=bash]
cat > custom.rules << "EOF"
Placement-rules:
/* charge on any atom */
atom: "*"            m 
/* charge + symmetric quarupole on bromine */ 
atom: "[Br]"         mqz
/* charge + dipole on p-hydrogen */
atom: "[#1]cccc[Br]" md
EOF
mult_fitter -Iphbr.mol2 -Gphbr.esp -p custom.rules \
            -Ophbr.custom.mmol
\end{lstlisting}
\lstinline{phbr.custom.mmol} contains the resulting molecule. The rulefile has a quite
simple syntax. We specify a SMARTS pattern of the desired atom (Note that hydrogens are
matched by \lstinline{"[#1]"} and NOT \lstinline{"[H]"}) and a list of the required 
multipoles: 
\lstinline{m}, \lstinline{d} and \lstinline{q} for a monopole, a dipole and a quadrupole 
respectively. The dipole and quadrupole may be modified with \lstinline{x}, \lstinline{y} 
or \lstinline{z} to enforce the symmetry along the selected axis. Additionaly, the dipole 
can be modified with a \lstinline{v} key followed by a vector, to direct the dipole along 
specified vector in the local coordinate frame. Also note, that in versions prior to 0.4.0
the dipoles were restricted along Z axis by default. To illustrate possible combinations 
we provide the following example which contains a list of valid rules:
\begin{lstlisting}[style=file]
Placement-rules:
/* charge on any atom */
atom: "*"            m 
/* charge + quarupole symmetric along Z axis */ 
atom: "[Br]"         mqz   
/* charge + unrestricted dipole on p-hydrogen */
atom: "[#1]cccc[Br]" md
/* charge + dipole along the (1,1,0) vector + quadrupole symmetric along Z axis */
/* note that *==z for quadrupoles, however the usage of asterisks is not recommended */
atom: "[S]" mdvq* (1,1,0)

\end{lstlisting}
The patterns are
matched in the same order as they go in the file. Each subsequent rule overrides
the previous ones, so the most general rules should go before anything special. When no
rulefile is specified the default, located in
\lstinline{__prefix__/share/}\lstinline{electrostatic_tools/}\lstinline{<version>/}
\lstinline{placement.rules},
is used.
\subsection{Topology Equivalence}
For this tutorial we will need the \lstinline{meoh.*} files from the \lstinline{examples}
directory.

By default \lstinline{mult_fitter} preserves topology equivalence so the equivalent atoms
get the equivalent charges and multipoles fitted. This behaviour can be overriden by a
\lstinline{-b} flag. For example we want to fit atomic charges for methanol molecule 
without respect to its
topology (namely the equivalence of hydrogen atoms within the $-CH_{3}$ group).
\begin{lstlisting}[language=bash]
mult_fitter -I meoh.mol2 -G meoh.esp -O meoh.b.mmol \
            -p m.rules -b
\end{lstlisting}
Note that we are using our custom multipole placement rules without any multipoles 
beyond monopole. The programs uses atomic charges as the source of topological information. 
If two atoms of the same chemical element have equal charges they are considered 
equivalent. 
By inspecting the \lstinline{meoh.mol2} and \lstinline{meoh.b.mmol} files one can check that
however the input file had a topologically symmetrical charges, the output does not.
On the contrary, if an input molecule has bad charges assigned, the topologically symmetric
Gasteiger charges may be recalculated with the \lstinline{-r} key to enforce the charge 
(and multipole) symmetry of the fit:
\begin{lstlisting}[language=bash]
mult_fitter -I meoh.mulliken.mol2 -G meoh.esp \
            -O meoh.r.mmol -p m.rules -r
\end{lstlisting}

\subsection{Extra-Point Position Fit}
The \lstinline{ep-fitter} program is used to perform the optimization of the extra-point (EP)
positions.
It places extra-points at atoms specified with SMARTS mask. We will call these atoms EP-hosts.
Lets perform the EP
position optimizaion for a \textit{m}-fluorobrombenzene which is saved in
\lstinline{mol36.mol2} and \lstinline{mol36.esp} files in the \lstinline{examples} directory.
We'll place the extra-points on the bromine atom and use a charge-only rulefile which can be 
found in the \lstinline{examples} directory.
\begin{lstlisting}[language=bash]
ep_fitter -I mol36.mol2 -G mol36.esp -p m.rules \
          -M"[Br]"
\end{lstlisting}
The output was saved to the \lstinline{mol36.ep.mmol} and \lstinline{mol36.ep.log} files. 
The program places the EP
on the local Z-axis of the atom, defined by the orient rules, and then searches for 
the position
with the lowest MEP RMSD value. The optimization tooks more time than a simple charge fit 
since it uses the Nelder-Mead simplex algorith for the position optimization with the 
charge refit at every step. The program saves the EP as the atom with a zero nuclear 
charge meaning a dummy, nonexistent, atom.

The use of the Nelder-Mead algorithm looks strange when dealing with a single parameter
optimization, however we can optimize several EP centers simultaneously. Lets try to 
add extra-points to the
both bromine and fluorine, This time we will need to increase a maximum iteration limit 
for the optimization to converge. We have also switched to a faster charge fitting 
algorithm to save some time. Additionally we specified a different file for the output.
\begin{lstlisting}[language=bash]
ep_fitter -I mol36.mol2 -G mol36.esp -p m.rules \
          -M"[Br]" -M"[F]" -a FullPivLU -m 150  \
          -O mol36.2ep.mmol
\end{lstlisting}
If you investigate the result you will notice, that the EP-Br distance converged to 1.3 \AA,
while the EP-F distance became -5 \AA. The negative value of the distance means that EP has
penetrated its host atom
and traversed inside the molecule. In this case, -5 \AA  actually means that it traversed 
the whole molecule and stopped at the reference grid border. This behaviour is normal 
for fluorines since they do not need any special anisotropy treatment.

One more thing should be mentioned about the EP position optimization possibilities. Since
we're using the same code for the charge fitting, it is possible to fit simultaneously both
multipoles and EP positions (with the relevant placement rules set). It is even possible 
to place multipoles on the extra-point centers and optimize their positions, however note, 
that internally EPs are stored as Einsteinium (\lstinline{"[#99]"}, because matching 
\lstinline{"[#0]"} doesnot work). When the multipole placement rules tell the fitter to 
add the multipoles on the EP-host atom 
it ignores this rule, leaving the host with charge only, however this behaviuor can be 
overriden by the \lstinline{-k} switch.

\subsection{Converting MMol to Common Format}
After we fitted the charges or multipoles, we can convert the resulting \lstinline{.mmol} 
file into any common chemical file format supported by OpenBabel with the
\lstinline{mmol2mol} program. Try this:
\begin{lstlisting}[language=bash]
mmol2mol phbr.mmol
\end{lstlisting}
By default it strips all multipoles and saves the molecule as a TRIPOS \lstinline{MOL2} 
file, but this behaviour can be overriden with the \lstinline{-O} and \lstinline{-t} flags 
to change the output file and its format if automatic extention based guessing fails. 
You can also try MCC conversion
options to preserve multipole data, however this part is still under research and is
subject to change.

\subsection{Multipole Charge Cluster (MCC) placement}
Actually the \lstinline{mmol2mol} is not a simple file format converter. It can not only 
strip (or add or copy) the multipoles present in the molecule, but also convert them to multipole charge 
cluster - a set of point charges, located very close to the host atom to simulate the original
multipoles. 
\lstinline{mmol2mol} adds from 2 to 6 extra-points per one host (depends on the multipole 
expansion symmetry), connected with single bonds to the central host atom. The MCC conversion
algorithms are turned on with a \lstinline{-M} key which specifies the SMARTS masks of atoms
which multipoles shoud be converted to MCCs, for example, if you already have a 
\lstinline{phbr.mmol} file from the \lstinline{mult_fitter} tutorial:
\begin{lstlisting}[language=bash]
mmol2mol -I phbr.mmol -M"[Br]" 
\end{lstlisting}
This command will create a \lstinline{phbr.mcc.mol2} file with the following contents
(note the increased atomic charge of bromine atom and two additional dummy atoms):
\begin{lstlisting}[style=file]
@<TRIPOS>MOLECULE
*****
 14 14 0 0 0
SMALL
GASTEIGER

@<TRIPOS>ATOM
      1 BR          0.9991    0.0803    0.1295 Br      1  LIG1      -45.8550
      2 C           2.8820    0.1014   -0.0402 C.ar    1  LIG1        0.2762
      3 C           3.5573    1.3195   -0.1026 C.ar    1  LIG1       -0.2648
      4 C           4.9466    1.3334   -0.2343 C.ar    1  LIG1       -0.1114
      5 C           5.6536    0.1320   -0.3014 C.ar    1  LIG1       -0.1564
      6 C           4.9736   -1.0845   -0.2350 C.ar    1  LIG1       -0.1114
      7 C           3.5842   -1.1018   -0.1044 C.ar    1  LIG1       -0.2648
      8 H           3.0135    2.2588   -0.0508 H       1  LIG1        0.1888
      9 H           5.4780    2.2803   -0.2855 H       1  LIG1        0.1465
     10 H           6.7358    0.1442   -0.4058 H       1  LIG1        0.1453
     11 H           5.5259   -2.0194   -0.2859 H       1  LIG1        0.1465
     12 H           3.0612   -2.0529   -0.0547 H       1  LIG1        0.1888
     13 XX          0.8995    0.0792    0.1385 Du      1  LIG1       22.8358
     14 XX          1.0987    0.0814    0.1205 Du      1  LIG1       22.8358
@<TRIPOS>BOND
     1     1     2    1
     2     2     3   ar
     3     3     4   ar
     4     4     5   ar
     5     5     6   ar
     6     6     7   ar
     7     2     7   ar
     8     3     8    1
     9     4     9    1
    10     5    10    1
    11     6    11    1
    12     7    12    1
    13     1    13    1
    14     1    14    1
\end{lstlisting}
Multiple \lstinline{-M} keys are possible. By default the MCC has the radius 
of 0.1 \AA, however it can be overriden with an 
\lstinline{-r} key. You can also tell the \lstinline{mmol2mol} to ignore some parts 
of multipole expansion with \lstinline{-d} and \lstinline{-q} keys for dipole and 
quadrupole (\lstinline{-dq} will strip all multipoles).

\subsection{Manually adding multipoles to the molecule}
The \lstinline{mmol2mol} can also place multipoles on atoms. This behaviour is turned on 
with \lstinline{-C}, \lstinline{-D} and \lstinline{-Q} keywords for monopole, dipole and 
quadrupole. The argument to this keywords are little complicated. You have provide a 
string surrounded by single (\lstinline{''}) or double (\lstinline{""}) quotes. This 
string should contain a SMARTS pattern of the host atom, followed by one or several 
values, specifying the multipole moment value (in a. u.). For atomic charge you'll 
obviuosly need a single value. The dipole can be specified with 3 values, or with a single
value, which is treated as $Z^{th}$ component of the dipole. The quadrupole can be specified 
as $Q_{zz}$ (for symmetric one), or as $Q_{xx}$, $Q_{yy}$ and $Q_{zz}$ triplet 
(however no checks are made for the zero-trace condition!), or as 
$Q_{xx}$, $Q_{yy}$, $Q_{zz}$, $Q_{xy}$, $Q_{xz}$ and $Q_{yz}$ set of values. For example, 
to add a quadrupole to the bromine atom in PhBr molecule issue:
\begin{lstlisting}[language=bash]
mmol2mol phbr.mol2 -O phbr.quad.mmol \
         -Q"[Br] 1.0 2.0 -3.0"
\end{lstlisting}
This command will create a file \lstinline{phbr.quad.mmol} with a quadrupole on 
bromine atom. You can also specify multiple number of \lstinline{-C}, 
\lstinline{-D} and \lstinline{-Q} keys.

You can also add multipoles and convert them to MCC simultaneously. \lstinline{mmol2mol}
first performs all multipole modifications requested, then converts the needed ones to MCC.
Try this command:
\begin{lstlisting}[language=bash]
mmol2mol phbr.mol2 -O phbr.manual.mol2 \
         -Q"[Br] 1.0 2.0 -3.0" -M"[Br]"
\end{lstlisting}

\subsection{Copying multipoles to the molecule}
There is another possibility to modify the multipoles with the \lstinline{mmol2mol} 
program. One can copy the multipoles from another file or files (useful for molecular 
complexes stored in single file) with a \lstinline{--copy-multipoles} key. The multipoles 
are copied directly with no checks applied. The only thing preserved is the order of atoms 
in the source and input file. The input file serves as the source of atoms and bonds, while 
the source file is used to get the multipoles. No multipoles in the input file are preserved.
You may use multiple \lstinline{--copy-multipoles} keys 
to copy from several sources. In this case the ordering of files also matters.

\subsection{Editing the .ESP files}
The \lstinline{Electrostatic Tools} also include an \lstinline{esp_modifier} program.
It's main purpose to modify the electrostatip potential maps by adding or removing 
potential generated by molecules. For example we want (1) to fit the multipoles for 
PhBr molecule, (2) fit the RESP charges without Br's quadrupole 
and then (3) add this quadrupole back to Br atom.

The first step is described above in the \lstinline{mult_fitter} tutorial, however we can 
repeat it here (This should display the fitted quadrupole of bromine atom):
\begin{lstlisting}[language=bash]
mult_fitter phbr.mol2 phbr.esp
grep 'Quadrupole' phbr.mmol
\end{lstlisting}
For the second step we need to prepare the \lstinline{.esp} file without quadrupole on Br.
The \lstinline{esp_modifier} can substract the potential of the whole molecule, so we need 
to prepare a special PhBr with no charges, just quadrupole:
\begin{lstlisting}[language=bash]
mmol2mol phbr.mmol -C"* 0" \
         -O phbr.q_only.mmol
\end{lstlisting}
Next, substract this molecule's potential from \lstinline{.esp} file:
\begin{lstlisting}[language=bash]
esp_modifier phbr.esp phbr.no_q.esp \
             -R phbr.q_only.mmol
\end{lstlisting}
The \lstinline{-R} flag removes molecule's potential from the \lstinline{.ESP} file, 
while \lstinline{-A} flag add the potential. Multiple \lstinline{-R} and \lstinline{-A} 
keys are possible in one command.

Now you can use your favourite RESP fitting tool to fit the charges for the molecule with 
the modified grid file. Assuming that you obtained a \lstinline{phbr.resp.mol2} file with 
the RESP charges, you can add the quadrupole back with:
\begin{lstlisting}[language=bash]
mmol2mol phbr.resp.mol2 -Q"[Br] $(grep Quadrupole phbr.mmol | awk '{print $11}')" \
         -O phbr.resp_with_q.mmol
\end{lstlisting}

\subsection{Coplex MEP RMSD estimations}
The \lstinline{Electrostatic Tools} also include an \lstinline{mep_rmsd} program.
This program is used for complex MEP error calculations. To get the MEP error issue
the following :
\begin{lstlisting}[language=bash]
mep_rmsd -m phbr.mmol -g phbr.esp
\end{lstlisting}
The result is pretty self explanatory:
\begin{lstlisting}[language=bash]
phbr.mmol : 0.62615 kcal/mol*e ; N = 4617
\end{lstlisting}
Instead of batch error estimation, you also can also get the MEP error in selected 
region of the molecule. This behaviuor is controlled with \lstinline{-M} key followed 
by a triplet of central atom SMARTS, minimum angle between any chemical bond and vector 
to MEP point (in degrees), and maximum distance to the MEP point. By default (if only 
a SMARTS is given), the angle is set to 90$^{\circ}$ and the distance is unlimited so 
the error is estimated inside the hemisphere around the central atom ''outside'' 
the molecule. 

In case multiple \lstinline{-M} keys are specified, the regions of error 
estimation are joined. The program also accepts multiple input and grid files for batch 
estimations. In this case the number of molecule files should correspond to the number 
of grid files.

The following example will estimate the error in the hemisphere with the radius of 4\r{A} around the halogen 
atom in phenylbromide. 
\begin{lstlisting}[language=bash]
mep_rmsd -m phbr.mmol -g phbr.esp \
         -M"[Br] 90 4"
\end{lstlisting}

\subsection{Using Script Bindings}
To analyse fit results you can also use script language bindings. Unfortunately currently 
we do not provide any reference documentation, so you'll have to look into source code for 
reference. We'll provide a simple example python script here. This script uses the 
\lstinline{phbr.mmol} file generated earlier in this tutorial.
\begin{lstlisting}[language=Python]
import openbabel
import ElectrostaticTools

# read molecule with multipole support from file
mol = ElectrostaticTools.GeneralMultipoledMolecule()
mol.readMe("phbr.mmol")

# save it as MOL2 
# note: GeneralMultipoledMolecule is OBMol subclass
#       so we can use any OpenBabel API
conv = openbabel.OBConversion()
conv.SetOutFormat("mol2")
conv.WriteFile(mol, "test.mol2")

# we can iterate other the atoms
for atom in openbabel.OBMolAtomIter(mol):
  print (atom.GetAtomicNum())


# print molecule in MMOL format
print (mol.toString())

# check if multipole expansion exists on the first 
# atom and print the monopole and quadrupole value
if (mol.hasMultipoles(1)) :
  mult = mol.GetRawMultipoles(1)
  if (mult.hasMonopole()) :
    print ("\nMonopole value: ", 
           mult.monopole().value())
  if (mult.hasQuadrupole()) :
    print ("\nMonopole value: ", 
           mult.quadrupole().value().toString())
  # or simply print the expansion as is
  print ("\nMultipoles on the first atom:")
  print (mult.toString())

  # or we can get oriented multipoles
  print ("\nOriented multipole on the first atom:")
  print (mol.GetMultipoles(1).toString())
  
# or we can get raw multipoles from atom
mult = ElectrostaticTools.Multipoles(
         mol.GetAtom(2).GetData("Multipoles")
       )
print ("\nMultipoles from atom:")
print (mult.toString())
\end{lstlisting}
  
\newpage
\section{Program Overview}
\subsection{General notes}
All programs included in the package are non-interactive command line tools, written
with UNIX philosophy in mind -- they are silent if everything is going fine, however some
noncritical messages can be logged. All tasks are
formulated with command line arguments. Thanks to the \lstinline{boost::program_options} 
library there are several ways to pass an argument to the program. The following 
invocations are identical:
\begin{lstlisting}[language=bash]
progname --long-option arg
progname --long-option=arg
progname -l arg
progname -larg
\end{lstlisting}
However some constructions are NOT available, such as:
\begin{lstlisting}[language=bash]
progname --no-long-option
progname --with-long-option
progname --disable-long-option
# etc ...
progname -l=arg
\end{lstlisting}
Some arguments are considered essential for the program execution and can be specified
without a key. In this case the order of such keyless options is important. For
example:
\begin{lstlisting}[language=bash]
mult_fitter molecule.mol2 grid.esp      
# is fine
mult_fitter molecule.mol2 grid.esp -f  
# is fine
mult_fitter -c 1e9 molecule.mol2 grid.esp   
# is fine
mult_fitter grid.esp molecule.mol2      
# will terminate with an error
\end{lstlisting}

\subsection{mult\_fitter}
The \lstinline{mult_fitter} program performs fitting of the atom-centered multipoles to 
the specified
reference molecular electrostatic potential (MEP). The MEP is specified in a form of AMBER 
\lstinline{.esp} file. The input molecule can be specified in any chemical format
understood by the OpenBabel library. The result is saved
into our plan-text \lstinline{.mmol} file format. Multipole positioning is controlled by 
SMARTS patterns which are read from a separate file. The multipoles are fixed in the 
orientation regarding the neighboring atoms which is controlled through another file. 
It is possible to constrain the topological
equivalence of atoms (which can be precisely controlled) and the symmetry of quadrupoles. 
Addition of dummy multipole centers in the geometrical center of atom groups specified 
by a SMARTS pattern is also possible.

Some of possible invocation patterns:
\begin{lstlisting}[language=bash]
mult_fitter <input> <MEP .esp> [output] [options]
mult_fitter -I <input> -G <MEP .esp> -O <output> \
            [options]
\end{lstlisting}

Generic options:
\begin{description}
\item[$--$version] print the program name and version and exit.
\item[$--$help] print a help message to console and exit.
\end{description}

Input control:
\begin{description}
\item[$-$I, $--$input $<arg>$ \textit{required}] an input molecule file. This parameter is
also read as the first argument without a key, so the key can be omitted.
\item[$-$t, $--$filetype $<arg>$] an input molecule file type specified as common extenson
for this filetype. Generally the filetype is guessed by the file extension 
(\lstinline{GUESS} option). This key overrides this guess.
\item[$-$G, $--$grid $<arg>$ \textit{required}] a reference MEP in the \lstinline{.esp} 
format. 
This parameter is also read as the second argument without a key, so the key can be 
omitted. Atomic coordinates in the \lstinline{.esp} files are ignored, but the total charge 
is used as a constraint in the fitting procedure. 
\item[$-$p, $--$placement-rules $<arg>$] multipole placement rules specified 
in special format.
The multipoles are placed according to SMARTS patterns. Note that OpenBabel SMARTS are
different from Daylight SMARTS (see http://openbabel.org/wiki/SMARTS). See the format
description for more info on how to setup custom rules.
\item[$-$o, $--$orient-rules $<arg>$] multipole orient rules. The multipoles are placed
according to SMARTS patterns. Note that OpenBabel SMARTS are 
different from Daylight SMARTS (see http://openbabel.org/wiki/SMARTS). See the format
description for more info on how to setup custom rules.
\end{description}

Output control:
\begin{description}
\item[$-$O, $--$output $<arg>$] an output molecule file. This parameter is 
also read as the third argument without a key, so the key can be omitted. 
The default name is generated by
replacing the last file extension from the input file name with \lstinline{mmol}. In order 
to save previous results, the program will terminate if the output file exists. The molecule
is saved in a custom plain-text format which was designed to support atomic multipoles. 
See the format description for more info.
\item[$-$L, $--$log $<arg>$] a log file. The default name is generated by 
replacing the last file extension from the input file name with the \lstinline{log}. 
In order to 
save previous results, the program will terminate if the log file exists. The log
contains important information about program workflow and some fit data. It is
saved as a plain-text.
\item[$-$f, $--$force-output] a switch to overwrite the output and log files if they are
already present. 
\end{description}

MEP fit control:
\begin{description}
\item[$-$a, $--$algorithm $<arg>$] a MEP fit algorithm selector. Valid values are: 
\lstinline{SVD}, \lstinline{LLT}, \lstinline{LDLT}, \lstinline{PartialPivLU}, 
\lstinline{FullPivLU}, \lstinline{HouseholderQR}, \lstinline{ColPivHouseholderQR},
\lstinline{FullPivHouseholderQR}. 
The \lstinline{SVD} algorithm uses the method published by Sigfridson
and Ryde (J. Comput. Chem. \textbf{1998}, \textit{19}(4), 377). It works with a model
matrix without raising it to the power of two, thus increasing the stability of the fit. 
The other methods deal with the squared model matrix and a Lagrange constraints. The
details about them can be found in Eigen3 documentation.
\item[$-$c, $--$cutoff $<arg>$] a condition value cutoff (the largest singular value divided
by the smallest one) used in the SVD pseudoinversion procedures to eliminate statistical 
noise. This parameter is used only when fitting with the SVD fitter. The default value is 
$10^7$. 
\item[$-$r, $--$recalculate-topology] a flag to force fitter to recalculate atomic
charges used to force the equivalency of the topologically equivalent atoms. The atoms are
forced to have equal charges and multipoles if they belong to the same chemical element
and have the equal partial charges assigned. If this flag is set, the Gasteiger charges are
calculated and used for this purpose since they are topologically symmetrical. Note that
sometimes there are not enouth digits in a charge field of the input molecule file and
non-equivalent charges have the same charge, for example \textit{meta-} and \textit{para-} 
hydrogens and sometimes carbons in monosubstituted benzenes in the \lstinline{MOL2} file 
format. Generally turning on this
flag is a good idea if you are not controlling your input charges manually.
\item[$-$b, $--$break-equivalency] a flag to disable the equivalency constraints. The only
constraints added with this flag is total charge and quadrupole symmetry constraints.
\item[$--$refit] a flag that changes default fit behaviuor. When this flag is set the
\lstinline{mult\_fitter} will fit the multipoles, specified in the placement rule file, 
while saving the other multipoles present in the molecule, but not optimized in the current 
run. Without this flag these extra multipoles will be set to zero. The flag allows to fit 
the multipoles with respect to the present charges, e.g. to add quadrupoles, optimal for 
the RESP charges.
\end{description}

\subsection{ep\_fitter}

The \lstinline{ep_fitter} program fits the extra-point (EP) charge positions and 
atom-centered multipoles with the specified
reference molecular electrostatic potential (MEP). The MEP is specified in a form of AMBER 
\lstinline{.esp} file. The input molecule can be specified in any chemical format
understood by the OpenBabel library. The result is saved
into our plan-text \lstinline{.mmol} file format. Multipole positioning is controlled by 
SMARTS patterns which are read from a separate file. The multipoles are fixed in the 
orientation regarding the neighboring atoms which is controlled through another file. 
It is possible to constrain the topological
equivalence of atoms (which can be precisely controlled) and the symmetry of quadrupoles. 
Addition of dummy multipole centers in the geometrical center of atom groups specified 
by a SMARTS pattern is also possible. The extra-points are added to the host atoms 
specified by SMARTS patterns.

This program uses the same code as the \lstinline{mult_fitter} for multipole and charge
fitting so most of the options are exactly the same, however all options are provided here 
for consistency. The nonlocal EP position optimization runs on top of the linear charge and
multipole fitting. The Nelder-Mead simplex search local optimization algorithm is used for
this purpose.

Some of the possible invocation patterns:
\begin{lstlisting}[language=bash]
ep_fitter <input> <MEP .esp> -M"[Cl,Br,I]" \
          [output] [options]
ep_fitter -I <input> -G <MEP .esp> -O <output> \
          -M"[Cl,Br,I]" [options]
\end{lstlisting}

Generic options:
\begin{description}
\item[$--$version] print the program name and version and exit.
\item[$--$help] print a help message to console and exit.
\end{description}

Input control:
\begin{description}
\item[$-$I, $--$input $<arg>$ \textit{required}] an input molecule file. This parameter is
also read as the first argument without a key, so the key can be omitted.
\item[$-$t, $--$filetype $<arg>$] an input molecule file type specified as common extenson
for this filetype. Generally the filetype is guessed by the file extension 
(\lstinline{GUESS} option). This key overrides this guess.
\item[$-$G, $--$grid $<arg>$ \textit{required}] a reference MEP in the \lstinline{.esp} 
format. 
This parameter is also read as the second argument without a key, so the key can be 
omitted. Atomic coordinates in the \lstinline{.esp} files are ignored, but the total charge 
is used as a constraint in the fitting procedure. 
\item[$-$p, $--$placement-rules $<arg>$] multipole placement rules specified 
in special format.
The multipoles are placed according to SMARTS patterns. Note that OpenBabel SMARTS are
different from Daylight SMARTS (see http://openbabel.org/wiki/SMARTS). See the format
description for more info on how to setup custom rules.
\item[$-$o, $--$orient-rules $<arg>$] multipole orient rules. The multipoles are placed
according to SMARTS patterns. Note that OpenBabel SMARTS are 
different from Daylight SMARTS (see http://openbabel.org/wiki/SMARTS). See the format
description for more info on how to setup custom rules.
\end{description}

Output control:
\begin{description}
\item[$-$O, $--$output $<arg>$] an output molecule file. This parameter is 
also read as the third argument without a key, so the key can be omitted. 
The default name is generated by
replacing the last file extension from the input file name with \lstinline{mmol}. In order 
to save previous results, the program will terminate if the output file exists. The molecule
is saved in a custom plain-text format which was designed to support atomic multipoles. 
See the format description for more info.
\item[$-$L, $--$log $<arg>$] a log file. The default name is generated by 
replacing the last file extension from the input file name with the \lstinline{log}. 
In order to 
save previous results, the program will terminate if the log file exists. The log
contains important information about program workflow and some fit data. It is
saved as a plain-text.
\item[$-$f, $--$force-output] a switch to overwrite the output and log files if they are
already present. 
\end{description}

MEP fit control:
\begin{description}
\item[$-$a, $--$algorithm $<arg>$] a MEP fit algorithm selector. Valid values are: 
\lstinline{SVD}, \lstinline{LLT}, \lstinline{LDLT}, \lstinline{PartialPivLU}, 
\lstinline{FullPivLU}, \lstinline{HouseholderQR}, \lstinline{ColPivHouseholderQR},
\lstinline{FullPivHouseholderQR}. 
The \lstinline{SVD} algorithm uses the method published by Sigfridson
and Ryde (J. Comput. Chem. \textbf{1998}, \textit{19}(4), 377). It works with a model
matrix without raising it to the power of two, thus increasing the stability of the fit. 
The other methods deal with the squared model matrix and a Lagrange constraints. The
details about them can be found in Eigen3 documentation.
\item[$-$c, $--$cutoff $<arg>$] a condition value cutoff (the largest singular value divided
by the smallest one) used in the SVD pseudoinversion procedures to eliminate statistical 
noise. This parameter is used only when fitting with the SVD fitter. The default value is 
$10^7$. 
\item[$-$r, $--$recalculate-topology] a flag to force fitter to recalculate atomic
charges used to force the equivalency of the topologically equivalent atoms. The atoms are
forced to have equal charges and multipoles if they belong to the same chemical element
and have the equal partial charges assigned. If this flag is set, the Gasteiger charges are
calculated and used for this purpose since they are topologically symmetrical. Note that
sometimes there are not enouth digits in a charge field of the input molecule file and
non-equivalent charges have the same charge, for example \textit{meta-} and \textit{para-} 
hydrogens and sometimes carbons in monosubstituted benzenes in the \lstinline{MOL2} file 
format. Generally turning on this
flag is a good idea if you are not controlling your input charges manually.
\item[$-$b, $--$break-equivalency] a flag to disable the equivalency constraints. The only
constraints added with this flag is total charge and quadrupole symmetry constraints.
\end{description}

EP position fit control:
\begin{description}
\item[$-$M, $--$host-mask $<arg>$] the SMARTS mask of the EP host atoms. The key is 
repeatable
so multiple masks can be specified. The EP is placed on the local Z-axis specified by the
orientation rules and it is ensured that the EP is located ''outside'' of the molecule.
 By default the multipoles, placed on the EP hosts are removed, since it's strange to have 
two anisotropic enhancements for a single atom.
\item[$-$x, $--$fixed-position $<arg>$] fixed EP distance from halogen atom in Angstroms. 
No optimization will be applied. Overrides any optimization option.
\item[$-$d, $--$init-position $<arg>$] the initial distance between EP and its host atom in
Angstroms for optimization procedure. The default value is 1.5 \AA.
\item[$-$s, $--$init-step $<arg>$] the initial step of EP-host distance in Angstroms for the
optimization procedure. Positive values mean the increase of the distance while the negative
ones mean the decrease. The default value is 0.1 \AA.
\item[$-$e, $--$precision $<arg>$] a required position precision in Angstroms. Short name
is abbreviated from the word ''epsilon''. Technically, this is a maximum simplex radius. The
default value is $10^{-3}$ \AA.
\item[$-$m, $--$max-steps $<arg>$] the maximum number of optimization steps. The search is
stopped at this point and the failure is reported in log. The default value is 100.
\item[$-$k, $--$keep-multipoles] a flag to override removal of multipoles from EP hosts.
\end{description}

\subsection{mmol2mol}
The \lstinline{mmol2mol} preforms conversion between \lstinline{mmol} format and the other 
common chemical formats with respect to atomic multipoles. It can also modify multipole 
values per request on the atoms specified and convert multipoles to multipole charge 
clusters (MCCs).

Some of possible invocation patterns:
\begin{lstlisting}[language=bash]
mmol2mol <options>
mmol2mol <in_mol> <out_mol> [options]
\end{lstlisting}

Generic options:
\begin{description}
\item[$--$version] print the program name and version and exit.
\item[$--$help] print help message to console and exit.
\end{description}

Input control:
\begin{description}
\item[$-$I, $--$input $<arg>$ \textit{required}] an input molecule file in 
the \lstinline{.mmol} format. This parameter is also read as the first argument without 
a key, so the key can be omitted.
\item[$--$input$-$type $<arg>$] an input molecule file type specified as common extenson
for this filetype. Generally the filetype is guessed by the file extension 
(\lstinline{GUESS} option). This key overrides this guess.
\item[$-$o, $--$orient-rules $<arg>$] multipole orient rules. The multipoles are placed
according to SMARTS patterns. Note that OpenBabel SMARTS are 
different from Daylight SMARTS (see http://openbabel.org/wiki/SMARTS). See the format
description for more info on how to setup custom rules.
\end{description}

Output control:
\begin{description}
\item[$-$O, $--$output $<arg>$] an output molecule file. This parameter is also read as the
second argument without a key, so the key can be omitted. The default name is generated by
replacing the last file extension from the input file name with ''\lstinline{mol2}'' 
and the molecule is
saved in a TRIPOS \lstinline{MOL2} file format. In order to save the previous results, 
the program will terminate if the output file exists.
\item[$-$t, $--$output-type $<arg>$] an output molecule filetype, specified as a common 
extenson for this filetype. Generally the filetype is guessed by the output file extension 
(''\lstinline{GUESS}'' option). This key overrides this guess.
\item[$-$f, $--$force-output] a switch to overwrite the output files if it is
already present.
\end{description}

Multipole modification control:
\begin{description}
\item[$--$copy-multipoles $<arg>$] a list of molecule files, used as a source for 
multipoles. When this flag is specified, the program will combine the input molecule file
(a source of atom types and coordinates) with the source of multipoles overriding the 
multipoles in the input molecule. Note, that this behaviuor is dependent on the order of 
the atoms only. No additional chackes are applied.
\item[$-$C, $--$set-charge $<arg>$] a pair of SMARTS mask and a charge value (a.u.) 
to assing on a given SMARTS pattern. Note that you must use quotes around this pair.
\item[$-$D, $--$set-dipole $<arg>$] a pair of SMARTS mask and a dipole value (a.u.)
to assing on a given SMARTS pattern. Note that you must use quotes around this pair. 
A dipole can be specified as three values for each vector component, 
or as a single $d_{z}$ value.
\item[$-$Q, $--$set-quadrupole $<arg>$] a pair of SMARTS mask and a quadrupole value (a.u.)
to assing on a given SMARTS pattern. Note that you must use quotes around this pair. 
A quadrupole can be specified as a six component vector for $Q_{xx}$, $Q_{yy}$, $Q_{zz}$,
$Q_{xy}$, $Q_{xz}$ and $Q_{yz}$, or as a three component vector for 
$Q_{xx}$, $Q_{yy}$ and $Q_{zz}$, or as a single $Q_{zz}$ value.
\item[$--$add-charge $<arg>$] the same as \lstinline{$--$set-charge} but adds the charge
value to the existing one instead of overriding it.
\item[$--$add-dipole $<arg>$] the same as \lstinline{$--$set-dipole} but adds the dipole
value to the existing one instead of overriding it.
\item[$--$add-quadrupole $<arg>$] the same as \lstinline{$--$set-quadrupole} but adds the 
quadrupole value to the existing one instead of overriding it.
\end{description}

Conversion control:

\begin{description}
\item[$-$M, $--$mcc-mask $<arg>$] the multipoles on the atoms matching this mask will 
be substituted
by a multipole charge cluster (MCC). This cluster is composed of a set of several 
extra-point charges (2-6) and in combination with the charge of the host atom creates 
a MEP distribution, analogous to the multipolar one. The key is repeatable so multiple 
masks can be specified.
\item[$-$r, $--$radius $<arg>$] the MCC radius in Angstroms as the distance between 
the host atom and the extra-points. The default is 0.1 \AA.
\item[$-$d, $--$ignore-dipole] do not convert atomic dipoles to the MCC.
\item[$-$d, $--$ignore-quadrupole] do not convert atomic quadrupoles to the MCC.
\end{description}

\subsection{esp\_modifier}
The \lstinline{esp_modifier} program converts \lstinline{.mmol} files to common chemical
formats with respect to the atomic multipoles in these files.

Some of possible invocation patterns:
\begin{lstlisting}[language=bash]
esp_modifier <options>
esp_modifier <in_esp> <out_esp> [options]
\end{lstlisting}

Generic options:
\begin{description}
\item[$--$version] print the program name and version and exit.
\item[$--$help] print help message to console and exit.
\end{description}

Input control:
\begin{description}
\item[$-$I, $--$input $<arg>$ \textit{required}] an input molecular electrostatic 
potential file in \lstinline{ESP} format.
This parameter is also read as the first argument without 
a key, so the key can be omitted.
\item[$-$A, $--$add $<arg>$] a molecule, which potential is to be added to the 
\lstinline{ESP} file. Multiple keys are possible.
\item[$-$R, $--$remove $<arg>$] a molecule, which potential is to be removed from
the \lstinline{ESP} file. Multiple keys are possible.
\item[$-$c, $--$coordinates $<arg>$] a molecule, which atomic coordinates are to 
replace ones in the \lstinline{ESP} file.
\end{description}

Output control:
\begin{description}
\item[$-$O, $--$output $<arg>$] an output \lstinline{ESP} file. This parameter is also read as the
second argument without a key, so the key can be omitted. 
\item[$-$f, $--$force-output] a switch to overwrite the output files if it is
already present.
\end{description}

\subsection{mep\_rmsd}
The \lstinline{mep_rmsd} program estimates the MEP reproduction error on the grid
or it's subset.

Some of possible invocation patterns:
\begin{lstlisting}[language=bash]
mep_rmsd <options>
\end{lstlisting}

Input control:
\begin{description}
\item[$-$m, $--$molecules $<arg>$ \textit{required}] input molecule files for MEP error 
estimation in any format. 
\item[$-$g, $--$grids $<arg>$ \textit{required}] input grid files for MEP error 
estimation in \lstinline{ESP} format. The number of grid files must match the number 
of molecule files. 
\item[$-$M, $--$mask $<arg>$] a mask used to create a grid subset. Consists of a 
triplet of a SMARTS pattern, a minimum angle (in degrees) between any of the bonds 
of the first SMARTS atom and the vector to the grid point, and 
a maximum distance from the first SMARTS atom and the grid point. Multiple masks
are possible. In case of multiple masks or multiple atoms matching a single mask 
the MEP error is estimated on a union of the corresponding subsets.
tem[$--$no-coordinates-check $<arg>$] a flag to prevent checking the equality of 
atomic coordinates in the grid and molecule files. Do not use it unless  
completely sure it is needed.
\end{description}

\newpage
\section{Helper scripts}
Since verson 0.4.0 we also provide a set of useful Python 
(Python3 but should be easily modified for Python2) scripts written with the help
of Electrostatic Tools API. The scripts are NOT installed with \lstinline{make install}
command and are located in the \lstinline{scripts} directory. 
Feel free to copy and use them.

\subsection{get\_eel.py}
The \lstinline{get_eel.py} script calculates intra- and intermolecular electrostatic 
interaction energies for molecules in \lstinline{mmol} format and prints them 
to the console (in kcal/mol). The intermolecular interactions are calculated as 
interaction of selected molecule with the field of other molecules. In case of correct 
execution it should print $2N$ real numbers, where $N$ is the number of molecule 
files supplied.

Some of possible invocation patterns:
\begin{lstlisting}[language=bash]
get_eel.py -M <molfile1> .. <molfileN> \
           <options>
\end{lstlisting}

Arguments:
\begin{description}
\item[$-$h, $--$help] print help message to console and exit.
\item[$-$I, $--$input $<arg(s)>$ \textit{required}] input molecule file(s). 
Unlimited number of files possible.
\item[$--$weight-1-2] weight for the scaling of the 1-2 intramolecular interactions. 
The default value is 0.
\item[$--$weight-1-3] weight for the scaling of the 1-3 intramolecular interactions. 
The default value is 0.
\item[$--$weight-1-2] weight for the scaling of the 1-4 intramolecular interactions. 
The default value is 1/1.2 (AMBER default).
\item[$--$weight-intra] weight for the scaling of all the intramolecular interactions
(including the 1-2, 1-3, and 1-4 ones). The default value is 1.
\item[$--$weight-inter] weight for the scaling of all the intermolecular interactions. 
The default value is 1.
\end{description}

\subsection{get\_multipoles.py}
The \lstinline{get_multipoles.py} script extracts multipoles for an atom specified by 
SMARTS pattern from a \lstinline{mmol} file. Although the mmol files are often can be 
parsed with standard console tools such as \lstinline{head}, \lstinline{tail}, 
\lstinline{grep}, \lstinline{awk}, etc. in case of complicated molecules actual SMARTS
matching can be very helpful.

Some of possible invocation patterns:
\begin{lstlisting}[language=bash]
get_multipoles.py -M <molfile1> .. \
           <molfileN> \
           <options>
\end{lstlisting}

Arguments:
\begin{description}
\item[$-$h, $--$help] print help message to console and exit.
\item[$-$I, $--$input $<arg(s)>$ \textit{required}] input molecule file(s). 
Unlimited number of files possible.
\item[$-$M, $--$mask $<arg>$ \textit{required}] SMARTS atom mask to select the 
atom of interest.
\item[$-$m, $--$monopole] print atomic charge.
\item[$-$d, $--$dipole] print atomic dipole vector.
\item[$--$dx] print the $d_x$ component of the atomic dipole.
\item[$--$dy] print the $d_y$ component of the atomic dipole.
\item[$--$dz] print the $d_z$ component of the atomic dipole.
\item[$-$q, $--$quadrupole] print atomic quadrupole matrix (9 elements).
\item[$--$qxx] print the $Q_xx$ component of the atomic quadrupole.
\item[$--$qyy] print the $Q_yy$ component of the atomic quadrupole.
\item[$--$qzz] print the $Q_zz$ component of the atomic quadrupole.
\item[$-$b, $--$fill-blanks $<arg>$] if no multipole found fill the blank space with 
the string specified (off by default; ''0'' if use without an argument). 
\item[$--$no-header] do not print the header.
\end{description}

\newpage
\section{File Formats}

\subsection{General Notes}
All the following file formats support C-style comments 
(\lstinline[language=C]{/* comment */}) as well as C++-style comments
(\lstinline[language=C]{// comment till the end of line })
so any additional information can be stored next to the data in an arbitrary format. The
comment parsers are not very smart so do not nest your comments. The spaces, newlines and
tabulation characters are ignored, so a fancy text alighning can be achieved.
We prefer to save SMARTS patterns in quotes. These quotes are required by the format, so we
can check that a pattern was specified and we're not reading something different.

The files are separated in sections. Every section starts with a header ending with 
a colon sign and ends with the beginning of the next section.

\subsection{Multipole Orient Rules}
Multipole orient rules controls the orientation of a local coordinate frame for each atom. 
The rules start with a common ''\lstinline{Orient-rules:}'' header and contain records of
individual rules. The records are passed from top to bottom, with the latter overriding the
former, so the ordering is important. The first one should be something general with 
very specific ones at the bottom of the list.
\begin{lstlisting}[style=file]
Orient-rules:
rule: z "*"                                             
rule: a "[!#99]~[!#99]"                                            
rule: b "[!#99]=[!#99]~[!#99]"                                   
rule: c "[!#99](~[!#99])~[!#99]"                                   
rule: d "[!#99]([!#99])([!#99])[!#99]"                          
rule: z "[!#99]([!#99])([!#99])([!#99])[!#99]"           
rule: e "[#99]1[!#99]-,=,@[!#99]1"                     
\end{lstlisting}
Each rule starts with the ''\lstinline{rule:}'' keyword, followed by a letter, followed 
by a SMARTS
pattern. The quotes around the SMARTS pattern are mandatory.  The order of atoms in SMARTS
is important in most cases. The first atom is a center of the multipole expansion and 
the local frame orign, the meaning of the others are determined by the letter, 
which encodes the rule type. Note that we use \lstinline{"[#99]"} internally as 
a dummy atom, because SMARTS like \lstinline{"[#0]"} or \lstinline{"[#200]"} do not work. 
That's why the orientation rule's SMARTS patterns look a little strange. 
Also note, that if your molecule contains Einsteimium (which is hardly the case),
you should temporarly change it to something different. 
\begin{description}
\item{\textbf{z}} Only the first atom is important. This rule means that we do not care 
about the local frame orientation. 
For example it is hard to pick sensible orientation for tetrary carbons. Identity
matrix is used as the coordinate transformation matrix.
\item{\textbf{a}} The first two atoms are important. This is the rule for linear nonconjugated
fragments such as monovalent atoms or alkynes. The Z-axis is directed from the second atom to
the first. The X- and Y- axes are undefined and picked through a vector product of
the local Z-axis and the global axes.
\item{\textbf{b}} The first three atoms are important. This is the rule for conjugated
linear fragments, or the fragments, where we can suspect any type of interaction with the nearest
neighbour, or for the atoms with double bonds. In this case the Z-axis is directed from 
the second atom to the first. The X-axis is
perpendicular to the plane, defined by the first three atoms in SMARTS and the Y-axis is
perpendicular to the local X- and Z- axes.
\item{\textbf{c}} The first three atoms are important. This rule is for bivalent linkers
like ether group. 
The Z-axis directed along the bisector of 2-1-3 angle and poits from the sharp end of this
angle ''outside'' of the molecule. The X-axis is
perpendicular to the plane, defined by the first three atoms of the SMARTS and the Y-axis is 
perpendicular to the X- and Z- axes.
\item{\textbf{f}} The first three atoms are important. This rule is for bivalent linkers
like ether group. The rule is similar to \textbf{c}, execept X- and Y- axes are rotated by 
45$^{\circ}$ .
\item{\textbf{d}} The first four atoms are important. This rule is for trivalent 
atoms like amine nitrogen. 
\begin{itemize}
\item In the case of a pyramidal configuration of the first atom, the Z-axis points out of 
the top of the pyramid in the
direction, formed as sum of normalized bond vectors, pointing to the top of the pyramid.
The X-axis is defined as a vector product of the Z-axis and the 2$\rightarrow$1 bond vector. 
The Y-axis is a vector product of the X- and Z- axes.
\item In the case of a planar configuration of the first atom, the X-axis is 
the 2$\rightarrow$1 bond vector, 
the Z-axis is a vector product of the X-axis and the 3$\rightarrow$1 bond vector, so 
it points out of the plane. 
The Y-axis is perpendicular to the both X- and Z- axes.
\end{itemize}
\item{\textbf{e}} The first three atoms are important. This is special rule for the dummies. 
The X-axis is defined as the vector from the second atom to the first one. The Z-axis 
is perpendicular to the
plane defined by the first three atoms. The Y-axis is perpendicular to the both X- and Z- axes.  
\end{description}

\subsection{Multipole Placement Rules}
The Multipole placement rules format is easy to read and modify. The rules start with a common 
''\lstinline{Placement-rules:}'' header. There are two types of records: 
''\lstinline{atom}'' and ''\lstinline{group}''.
The records are passed from the top to the bottom, with the latter overriding the former, so
the ordering is important. The first one should be something general with very specific ones at 
the
bottom of the list. See the example with an ''any atom'', followed by a ''heavy halogen'', 
followed by the ''heavy halogen, connected to an aromatic moiety'', etc. 
\begin{lstlisting}[style=file]
Placement-rules:

atom: "*"           m
atom: "[Cl,Br,I]"   mdzqz
atom: "[Cl,Br,I]a"  mqz
atom: "[s]"         mdvqz (1,1,0)

group: "c1ccccc1"   mdzqz
\end{lstlisting}
Each atom rule starts with a ''\lstinline{atom:}'' keyword, followed by a SMARTS pattern 
in quotes, followed by multipole flags. The quotes around the SMARTS pattern are mandatory.
The multipole flags can be any combination of ''\lstinline{m}'', ''\lstinline{d}'', and
''\lstinline{q}'', meaning a monopole, a dipole, and a quadrupole repectively.
The symmetry of the dipoles and quadrupoles may be restricted with the ''\lstinline{x}'', 
''\lstinline{y}'', or ''\lstinline{z}'', keyword following the corresponding multipole 
keyword.\footnote{Note that in versions prior to 0.4.0 the dipoles were aligned with 
Z-axis by default. After 0.4.0 this behaviour has changed. So ''\lstinline{d}'' in 
pre-0.4 equals to ''\lstinline{dz}'' in 0.4.0 and later versions. } 
This results in dipoles being aligned with the corresponding local coordinate 
frame axis and quadrupole having it's main symmetry component aligned with this axis. 
Additionally, the dipoles can be forced to align with an arbitrary vector with the 
''\lstinline{v}'' keyword followed by a direction vector, expressed in local coordinate 
frame (see the example with aromatic sulfur atom).
When a SMARTS match of an atom rule happens, Electrostatic Tools programs will 
add the specified multipoles to the first atom of the SMARTS pattern. 

Group rules start with ''\lstinline{group:}'' keyword, followed by a SMARTS pattern 
in quotes, followed by the multipole flags. The formatting and properties are analogous the 
to atom rules. When a group SMARTS match happens, a program will add a dummy atom center 
to the geometrical
center of all SMARTS atoms and place the specified multipoles on this dummy
center. The dummy center becomes connected with the first two atoms in the group's SMARTS.

\subsection{Multipole Molecule}
The ''\lstinline{mmol}'' format was designed to store molecules with associated multipoles. 
It consists of the two sections: the molecule with atoms and bonds, and the multipole 
orient rules part.
The latter is described in the corresponding section above. The molecule section contains
''Atom'' and ''Bond'' records.
\begin{lstlisting}[style=file]
Molecule:
Atom: 6 ( 1.1057, 0.0178, -0.0171 )
Multipoles: ( 1.1057, 0.0178, -0.0171 )
  Monopole: -0.136

Atom: 8 ( 2.5213, 0.0064, -0.0264 )
Multipoles: ( 2.5213, 0.0064, -0.0264 )
  Monopole: -0.26592
  Dipole: ( 0, 0, -0.4019 )
  Quadrupole: ( -0.82922, 0, 0, 0, 1.0774, 0, 0, 0, -0.24822 )

Atom: 1 ( 0.7455, 0.9809, -0.3871 )
Multipoles: ( 0.7455, 0.9809, -0.3871 )
  Monopole: 0.062576

Atom: 1 ( 0.7455, -0.1514, 1.0007 )
Multipoles: ( 0.7455, -0.1514, 1.0007 )
  Monopole: 0.062576

Atom: 1 ( 0.7398, -0.7799, -0.6679 )
Multipoles: ( 0.7398, -0.7799, -0.6679 )
  Monopole: 0.062576

Atom: 1 ( 2.8166, 0.7242, 0.5592 )
Multipoles: ( 2.8166, 0.7242, 0.5592 )
  Monopole: 0.2142

Bond: 5 - 1 : 1
Bond: 3 - 1 : 1
Bond: 2 - 1 : 1
Bond: 2 - 6 : 1
Bond: 1 - 4 : 1

Orient-rules:
rule:   z       "*"
rule:   a       "[!#99]~[!#99]"
rule:   c       "[!#99](~[!#99])~[!#99]"
\end{lstlisting} 
The atom record starts with the ''\lstinline{Atom:}'' keyword followed by a nuclear charge 
(a. u.) and nuclear coordinates in brackets, separated by a comma (in Angstroms). 
Optionally it can contain a ''\lstinline{Multipoles}'' field.

The ''Multipoles'' record starts with the ''\lstinline{Multipoles:}'' keyword followed 
by the coordinates of the expansion center
in brackets, separated by a comma (in Angstroms). Next, it contains 
three optional fields: a ''\lstinline{Monopole:}'', a ''\lstinline{Dipole:}'' and a
''\lstinline{Quadrupole:}'' records, followed by the corresponding multipole moment 
value (in a. u.). Tensor values are written in brackets with components separated by 
a comma. A single ''Multipoles'' record has a single internal coordinate system. 
The multipoles are written in terms of the principal axes of the quadrupole. 
The orientation of the local coordinate frame is governed by the orient rules,
recorded in its own section of file. We use the following formulae to calculate the
electrostatic potential from the multipoles:

\[ V(\textbf{\overrightarrow{r}}) = \frac{q}{|\textbf{\overrightarrow{r}}|} + 
   \frac{\textbf{\overrightarrow{r}}\textbf{\overrightarrow{d}}}
        {|\textbf{\overrightarrow{r}}|^3} +
 \frac{\textbf{\overrightarrow{r}}\textbf{Q}\textbf{\overrightarrow{r}}}
        {|\textbf{\overrightarrow{r}}|^5}   \]
, where \(\textbf{\overrightarrow{r}}\) corresponds to radius-vector from the center of
the multipole expansion to a potential estimation point, \(q\), \(\textbf{\overrightarrow{d}}\)
and \(\textbf{Q}\) correspond to the charge, the dipole moment vector and the quadrupole 
moment matrix, transformed to the global coodinate frame.

The bond record starts with the ''\lstinline{Bond:}'' keyword, followed by the first 
atom index, followed by a ''minus'' sign, followed by the second atom index, 
followed by a colon sign, followed by a bond order value: 
(''\lstinline{1}'', ''\lstinline{2}'', ''\lstinline{3}'' or ''\lstinline{a}'' 
for single, double, triple aromatic bonds respectively). 

\end{document} 
